\section{Related work}\label{sec:related}

\todo{Paiva}
\todo{Segmentation}

Melody transcription is often cast as a melody \emph{extraction} task, which requires us to annotate an audio recording with a time-varying, continuous fundamental frequency (F0) estimate \cite{goto2004real}. Melody extraction has been popularized by annual results of the MIREX Audio Melody Extraction tasks \cite{downie2014ten}, garnering significant interest from the MIR community over the last two decades \cite{salamon2014melody,rao2022melody}. In contrast, our work is concerned with transcription to human-readable \emph{lead sheets}: this task requires us to indicate the positions of notes within a metrical structure, belonging to discrete pitch classes.

Lead-sheet melody transcription has received considerably less attention. Poliner et. al. observed in 2007 that ``an attempt was made to evaluate the lead voice transcription at the lowest level of abstraction [melody extraction], and as such, the concept of segmenting the fundamental frequency predictions into notes has been largely omitted from consideration'' \cite{poliner2007melody}. This omission largely remains true today. To the best of our knowledge, there have been only three efforts to transcribe melodies to lead sheets since 2007 \cite{ryynanen2008automatic,weil2009automatic,laaksonen2014automatic}; we compare our melody transcription results with \cite{ryynanen2008automatic} in \cref{sec:experiments}.

Polyphonic music transcription has received substantial attention, with its own MIREX task (Multiple Fundamental F0 Estimation) and a growing collection of supervised training data resources \cite{benetos2013automatic,thickstun2017learning,hawthorne2019enabling,manilow2019cutting}. Like melody transcription, polyphonic transcription is typically formalized as a low-level \emph{frame-based} transcription task, which requires us to annotate an audio recording with a time-aligned piano roll. Like melody extraction, this task does not result in a human-readable lead sheet or score. Nevertheless, the similarity of the polyphonic and melody transcription problems, together with recent progress towards polyphonic transcription, motivate us to consider whether representations learned by a polyphonic music transcription system, MT3 \cite{gardner2021mt3}, are useful for melody transcription.

% CHRIS: This maps cleanly onto our contributions (standardized evaluation and new datasets), but may be too snarky for paragraph 2
%Finally, we argue that a historical focus on the important but disparate task of melody \emph{extraction}---detecting the time-varying fundamental frequency of the melody as opposed to its discrete notes---has led to a lack of work on transcription and systemic issues in evaluation.